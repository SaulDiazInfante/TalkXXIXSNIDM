{
\begin{frame}{Control policies}
    \begin{textblock*}{55mm}(2mm, 10mm)
        \begin{beamerboxesrounded}{Optimal control theory is a way.}
            \begin{itemize}
                \item
                    We require a \textbf{model} to describe the spreading of an 
                    uncontrolled disease, and whose transitions generate a 
                    \textbf{cost or reward}.
                \item
                    Continuous \textbf{control action} that modify changes 
                    between states.
                \item
                    A \textbf{functional} which describes \textbf{cost-reward}.
            \end{itemize}
        \end{beamerboxesrounded}
    \end{textblock*}
%
    \begin{textblock*}{65mm}(62mm, 10mm)
        \begin{beamerboxesrounded}{Optimal control theory is a way.}
            \begin{itemize}
                \item
                    A rule that prescribes which control
                    operation to use at each time, is a \textbf{control policy}.
                \item
                    \textbf{closed-loop} or \textbf{feedback} control.
                \item
                    \textbf{open-loop} policy.
            \end{itemize}
            %
            We consider control policies that affect the bounded rates at which 
            population moves from one class (e.g., infected) to another (e.g., 
            recovered).
        \end{beamerboxesrounded}
    \end{textblock*}
    %
    %
    \begin{textblock*}{90mm}(25mm, 65mm)
        \begin{bibunit}[apalike]
            \nocite{Wickwire1977}
            \putbib
        \end{bibunit}
    \end{textblock*}
\end{frame}
}